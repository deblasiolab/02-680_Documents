
\documentclass[11pt, oneside]{article}   	% use "amsart" instead of "article" for AMSLaTeX format

\usepackage{geometry}                		% See geometry.pdf to learn the layout options. There are lots.
\geometry{letterpaper}                   		% ... or a4paper or a5paper or ... 
%\geometry{landscape}                		% Activate for rotated page geometry
\usepackage[parfill]{parskip}    		% Activate to begin paragraphs with an empty line rather than an indent
\usepackage{graphicx}				% Use pdf, png, jpg, or eps§ with pdflatex; use eps in DVI mode
								% TeX will automatically convert eps --> pdf in pdflatex		
\usepackage{amssymb}
\usepackage{amsmath}
\usepackage{tablefootnote}
\usepackage{multirow}
\usepackage{tabularx}
\renewcommand\tabularxcolumn[1]{m{#1}}

\usepackage{dsfont}
\usepackage[hang,flushmargin]{footmisc}

\usepackage{pgfplots,pgfplotstable}

\makeatletter
\long\def\ifnodedefined#1#2#3{%
    \@ifundefined{pgf@sh@ns@#1}{#3}{#2}%
}

\pgfplotsset{
    discontinuous/.style={
    scatter,
    scatter/@pre marker code/.code={
        \ifnodedefined{marker}{
            \pgfpointdiff{\pgfpointanchor{marker}{center}}%
             {\pgfpoint{0}{0}}%
             \ifdim\pgf@y>0pt
                \tikzset{options/.style={mark=*, fill=white}}
                %\draw [densely dashed] (marker-|0,0) -- (0,0);
                \draw plot [mark=*] coordinates {(marker-|0,0)};
             \else
                \tikzset{options/.style={mark=none}}
             \fi
        }{
            \tikzset{options/.style={mark=none}}        
        }
        \coordinate (marker) at (0,0);
        \begin{scope}[options]
    },
    scatter/@post marker code/.code={\end{scope}}
    }
}

\newcommand*\Eval[3]{\left(#1\middle)\right\rvert_{#2}^{#3}}


\usepackage{arydshln}
\usepackage{mathtools}

%Accessibility issues
\usepackage[a-1b]{pdfx}

%\usepackage{tagpdf}
%\tagpdfsetup{activate,paratagging,interwordspace}
\usepackage{accsupp}
\usepackage{hyperref}
\usepackage{lipsum}
%\usepackage{axessibility}
\usepackage{accessibility}

%SetFonts
\DeclareEmphSequence{\bfseries\itshape}
\DeclareMathOperator {\ints}{\mathbb{Z}}
\DeclareMathOperator {\reals}{\mathbb{R}}
\DeclareMathOperator*{\argmin}{argmin}
\DeclareMathOperator*{\argmax}{argmax}


\renewcommand\labelenumi{(\theenumi)}
\renewcommand*{\thefootnote}{\fnsymbol{footnote}}

\usepackage{tcolorbox}

\newenvironment{aside}[1][Aside]{\begin{tcolorbox}[colback=black!5,colframe=black!75!black,title=#1]}{\end{tcolorbox}}


\title{Topic 17: Expectation and Moments}
\author{02-680: Essentials of Mathematics and Statistics}
%\date{}							% Activate to display a given date or no date

\begin{document}
\maketitle

%%%%%%%%%%%%%%
\section{Expectation}
Given a probability distribution $p(X)$ of a random variable $X$, how can we summarize the distribution with a single value? 

The expectation of a random variable is a number that attempts to capture the center of $p(X)$ 
and can be interpreted as the long-run average of many independent samples from the given distribution.

For a (discrete) random variable $X$, the \emph{expected value} is 
\[\E[X] = \sum_{x\in range(X)} x \cdot p(X=x).\]
We sometimes also call this the \emph{mean} or \emph{first moment} of the variable. 

For instance, in our previous examples, for a pair of 4-sided die: 
\[\E[C] = 2 \cdot \frac{1}{16} + 3  \cdot \frac{2}{16} + 4 \cdot\frac{3}{16} + 5\cdot\frac{4}{16} + 6\cdot\frac{3}{16} + 7\cdot\frac{2}{16} + 8 \cdot \frac{1}{16} = \frac{80}{16} = 5\]

Note that in this case the expected value happens to be one of the outcomes, 
but if we consider a random variable $D$ which is the outcome of a single 6-sided die: 
\[\E[D] = 1 \cdot \frac{1}{6} + 2 \cdot \frac{1}{6} + 3 \cdot \frac{1}{6} + 4 \cdot \frac{1}{6} + 5 \cdot \frac{1}{6} + 6 \cdot \frac{1}{6} = \frac{21}{6} = \frac{7}{2}\]
which is not a possible outcome. 

For a continuous random variable, again we swap a sum for an integral:
\[\E[X] = \int_{-\infty}^\infty x \cdot f(x) dx\]

So if we look at random variable $U\sim Uniform(-1,3)$, and thus $f(u)=\frac{1}{4}$. 
\[\E[U] = \int_{-\infty}^\infty x \cdot \frac{1}{4} dx = \left.\frac{1}{8} x^2 \right|_{-1}^3 = \frac{9}{8} - \frac{1}{8} = 1\]

%%%%%%%%%%%%%%
\section{Variance and Standard Deviation}
In addition to the mean, many times we want to know something about how ``far'' the likely outcomes are from the mean,
in the normal distribution from last week this visually is about how narrow the bell is. 
To do this we define the \emph{variance} of a random variable as  
\[\V[X] = \E\left[\left[X-\E\left[X\right]\right]^2\right]\]
The variance is the expected squared difference between a random variables and its mean. 

The \emph{standard deviation} of a variable is the square root of the variance:
\[\sigma(X) = \sqrt{\V[X]} = \sqrt{ \E\left[\left[X-\E\left[X\right]\right]^2\right]}\]

If we look at a first coin, $F$:
\[\E[F] = \frac{1}{2}\]
The difference between $X$ and $\E[F]$ is $+\frac{1}{2}$ and $-\frac{1}{2}$ (with equal probability). 
\[\V[F] = \E\left[\left[F-\E\left[F\right]\right]^2\right] = \left(\frac{1}{2}\left(+\frac{1}{2}\right)^2 + \frac{1}{2}\left(-\frac{1}{2}\right)^2\right) = \frac{1}{4}\]
and thus
\[\sigma[F] = \sqrt{\V[F]} = \sqrt{\frac{1}{4}} =\frac{1}{2}.\]

What if instead we tried to simulate a coin flip into variable $S$. 
But we made a mistake and now we get 1 or 0 with probability $0.4995$, but in some cases we output 100 (with probability $0.001$). 
What are $\E[S], \V[S], \sigma[S]$?
\[\E[S] = 0 \cdot 0.4995 + 1 \cdot 0.4995 + 100 * 0.001 = 0 + 0.4995 + 0.1 = 0.5995\]
\[\V[S] = \E\left[\left[S-\E\left[S\right]\right]^2\right] \]\[ = \left(0.4995 \cdot 0.5995^2\right) + \left(0.4995 \cdot 0.4005^2\right) + \left(0.01 \cdot  99.4005^2\right) 
\approx  0.180 + 0.080 + 9.880 \approx 10.14\]
\[\sigma[S] = \sqrt{\V[S]} \approx \sqrt{10.14} \approx 3.184\]
So clearly the variance is much higher, but how often do we actually expect to see a value of 100? 
If $p(S=100)=0.001$ thats approximately one in every 1000 throws,
if we throw much less than that we may think its a fair coin. 
That leads the question of how we measure these things in practice. 

%%%%%%%%%%%%%%%
\section{Properties of Expectation and Variance}

%%%%%
\subsection{Summation}
For some set $X_1, X_2, ..., X_n$ of independent random variables
\[\E\left[\sum_{i\in[n]} X_i\right] = \sum_{i\in[n]} \E[X_i] \hspace{5em} 
\V\left[\sum_{i\in[n]} X_i\right] = \sum_{i\in[n]} \V[X_i]\]

%%%%%
\subsection{Linearity}
For some random variable $X$ and constants $a,b\in\reals$
\[\E\left[aX+b\right] = a\E[X]+b \hspace{5em} 
\V\left[aX+b\right] = a^2\V[X]\]

%%%%%%%%%%%%%%%%%
\section{Conditional Expectation and Variance}

%%%%%%%%%%%
\subsection{Discrete}
\[\E\left[X\mid Y=y\right] = \sum_x \left(x \cdot p(X=x \mid Y=y)\right)\]
\[\V\left[X\mid Y=y\right] = \sum_x \left(\left[x-\E\left[X\mid Y=y\right]\right]^2 \cdot p(X=x \mid Y=y)\right)\]

%%%%%%%%%%%
\subsection{Continuous}
\[\E\left[X\mid Y=y\right] = \int_{-\infty}^\infty \left(x \cdot f(x\mid y) dx\right)\]
\[\E\left[X\mid Y=y\right] = \int_{-\infty}^\infty \left(\left[x-\E\left[X\mid Y=y\right]\right]^2 \cdot f(x\mid y) dx\right)\]

\paragraph{Example.}
Assume $X\sim Uniform(0,1), Y\mid X=x \sim Uniform(x,1)$. 
\begin{align*}
\E[Y\mid X=x] &= \int_{-\infty}^\infty \left(y \cdot f(y\mid x) dy\right) \\
&= \int_x^1 \left(y \cdot \frac{1}{1-x} dy\right)\\
&= \frac{1}{1-x} \int_x^1 \left(y dy\right)\\
&= \frac{1}{1-x} \cdot \left[\frac{1}{2}y^2\right]^1_x\\
&= \frac{1}{1-x} \cdot \left[\frac{1}{2}-\frac{1}{2}x^2\right]\\
&= \frac{1-x^2}{2(1-x)} = \frac{(1+x)}{2}
\end{align*}
\begin{align*}
\V[Y\mid X=x] &= \int_{-\infty}^\infty  \left(\left[y-\E\left[Y\mid X=x\right]\right]^2 \cdot f(y\mid x) \right)dy \\
& =  \int_x^1  \left(\left[y - \frac{(1+x)}{2}\right]^2 \cdot  \frac{1}{1-x} \right)dy\\
& =  \frac{1}{1-x} \int_x^1  \left(\frac{\left(2y - (1 +x)\right)^2}{4} \right)dy\\
& = \frac{1}{1-x} \int_x^1  \left(\frac{\left(4y^2 - 4(1 +x)y + (1 +x)^2\right)}{4} \right)dy\\
& = \frac{1}{1-x} \int_x^1  \left(y^2 - (1+x)y +\frac{(1 +x)^2}{4} \right)dy\\
& = \frac{1}{1-x} \left[\frac{y^3}{3} - \frac{(1+x)}{2}y^2 +\frac{(1 +x)^2}{4}y \right]^1_x\\
& = \frac{1}{1-x} \left[\frac{y^3}{3} + \frac{y(1+x)((1+x)-2y)}{4} \right]^1_x\\
& = \frac{1}{1-x} \left[\frac{1-x^3}{3} + \frac{(1+x)^2(x-1))}{4} \right]^1_x\\
& = \frac{1}{1-x} \left[ \frac{(1-x)(1+x+x^2)}{3} - \frac{(1+x)^2\left(1-x\right)}{4} \right]\\
& = \frac{(1+x+x^2)}{3} -\frac{(1+x)^2}{4}
\end{align*}

%%%%%%%%%%%%%%%%%
\section{Covariance \& Correlation}
Given two random variables $X$ and $Y$ with expectations 
$\mu_X = \E[X]$ and $\mu_Y=\E[Y]$ respectively. 
The \emph{covariance} between the two is defined as
\[Cov(X,Y) = \E\left[(X-\mu_x)(Y-\mu_Y)\right]=\E[XY]-\E[X]\E[Y]\]

Assume also that the standard deviations are $\sigma_X$ and $\sigma_Y$, 
then the \emph{correlation} of $X$ and $Y$ is defined as 
\[\rho(X,Y) = \frac{Cov(X,Y)}{\sigma_X\sigma_Y}\]

You can think of correlation as a normalized version of covariance. 

\end{document}