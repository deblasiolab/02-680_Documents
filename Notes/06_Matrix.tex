\documentclass[11pt, oneside]{article}   	% use "amsart" instead of "article" for AMSLaTeX format

\usepackage{geometry}                		% See geometry.pdf to learn the layout options. There are lots.
\geometry{letterpaper}                   		% ... or a4paper or a5paper or ... 
%\geometry{landscape}                		% Activate for rotated page geometry
\usepackage[parfill]{parskip}    		% Activate to begin paragraphs with an empty line rather than an indent
\usepackage{graphicx}				% Use pdf, png, jpg, or eps§ with pdflatex; use eps in DVI mode
								% TeX will automatically convert eps --> pdf in pdflatex		
\usepackage{amssymb}
\usepackage{amsmath}
\usepackage{tablefootnote}
\usepackage{multirow}
\usepackage{tabularx}
\renewcommand\tabularxcolumn[1]{m{#1}}

\usepackage{dsfont}
\usepackage[hang,flushmargin]{footmisc}

\usepackage{pgfplots,pgfplotstable}

\makeatletter
\long\def\ifnodedefined#1#2#3{%
    \@ifundefined{pgf@sh@ns@#1}{#3}{#2}%
}

\pgfplotsset{
    discontinuous/.style={
    scatter,
    scatter/@pre marker code/.code={
        \ifnodedefined{marker}{
            \pgfpointdiff{\pgfpointanchor{marker}{center}}%
             {\pgfpoint{0}{0}}%
             \ifdim\pgf@y>0pt
                \tikzset{options/.style={mark=*, fill=white}}
                %\draw [densely dashed] (marker-|0,0) -- (0,0);
                \draw plot [mark=*] coordinates {(marker-|0,0)};
             \else
                \tikzset{options/.style={mark=none}}
             \fi
        }{
            \tikzset{options/.style={mark=none}}        
        }
        \coordinate (marker) at (0,0);
        \begin{scope}[options]
    },
    scatter/@post marker code/.code={\end{scope}}
    }
}

\newcommand*\Eval[3]{\left(#1\middle)\right\rvert_{#2}^{#3}}


\usepackage{arydshln}
\usepackage{mathtools}

%Accessibility issues
\usepackage[a-1b]{pdfx}

%\usepackage{tagpdf}
%\tagpdfsetup{activate,paratagging,interwordspace}
\usepackage{accsupp}
\usepackage{hyperref}
\usepackage{lipsum}
%\usepackage{axessibility}
\usepackage{accessibility}

%SetFonts
\DeclareEmphSequence{\bfseries\itshape}
\DeclareMathOperator {\ints}{\mathbb{Z}}
\DeclareMathOperator {\reals}{\mathbb{R}}
\DeclareMathOperator*{\argmin}{argmin}
\DeclareMathOperator*{\argmax}{argmax}


\renewcommand\labelenumi{(\theenumi)}
\renewcommand*{\thefootnote}{\fnsymbol{footnote}}

\usepackage{tcolorbox}

\newenvironment{aside}[1][Aside]{\begin{tcolorbox}[colback=black!5,colframe=black!75!black,title=#1]}{\end{tcolorbox}}

%SetFonts


\title{Topic 6: Matrices}
\author{02-680: Essentials of Mathematics and Statistics}
%\date{}							% Activate to display a given date or no date

\begin{document}
\maketitle

%%%%%
You can almost think of a \emph{matrix} as a 2-dimension vector. 
We say that an ``$n$-by-$m$'' matrix $M \in \reals^{n\times m}$ has $n$ rows and $m$ columns and we usually write it as:
\[
M = \left[\begin{matrix}
M_{1,1}& 	M_{1,2}& 	\dots& M_{1,m}\\
M_{2,1}& 	M_{2,2}& 	\dots& M_{2,m}\\ 
\vdots & \vdots & \ddots & \vdots \\ 
M_{n,1}& 	M_{n,2}& 	\dots& M_{n,m}
\end{matrix}\right]
\]



%%%%%%%%%
\section{Simple Matrix Operations}

\subsection{Addition and Scalar Multiplication.}
Like with vectors, addition of two matrices as well as scalar multiplication are element-wise operations, so for matrices $M,N \in \reals^{n\times m}$ and scalar $a\in\reals$:
\[O = M+N \rightarrow O_{i,j} = M_{i,j} + N_{i,j} \;\; \forall 1 \le i \le n, 1 \le j \le m\]
\[O = aM \rightarrow O_{i,j} = a M_{i,j} \;\; \forall 1 \le i \le n, 1 \le j \le m\]

\subsection{Transpose} 
For a given matrix $M \in \reals^{n\times m}$, the transpose $M^T \in \reals^{m \times n}$ is defined such that:
\[
\forall I\in[0,n-1], j\in[0,m-1] : M^T_{j,i} = M_{i,j} 
\]
This operation works for both matrixes and vectors (which are really $n\times1$ matrices).
Some examples: 
\begin{center}
$\left[\begin{matrix}
1 & 2 & 3\\
4 & 5 & 6
\end{matrix}\right]^T = 
\left[\begin{matrix}
1 & 4 \\
2 & 5 \\
3 & 6
\end{matrix}\right]$%
\hspace{5em}
$\left[\begin{matrix}
7 \\
8 \\
9 \\
10 \\
\end{matrix}\right]^T = \left[\begin{matrix}
7 & 8 & 9 & 10
\end{matrix}\right]
$
\end{center}
%%%%%%%%%
\section{Matrix Multiplication}
Just like with vectors, multiplying two matrices is more complicated than scalars. 
The first question is the size of the result, if we multiply $C \in \reals^{n\times p}$ with $D \in \reals^{p\times m}$ we get a matrix $E \in \reals^{n\times m}$;
notice that the \textit{inner} dimensions are the same.
And the values in $E$ are defined as follows:
\[
E_{i,j} = \sum_{k=1}^m C_{i,k}D{k,j}
\]

We can actually rewrite this using dot product, lets say that $C_{i,*}$ is the $i$-th column of $C$, and $D_{*,j}$ is the $j$-th column of $D$.
In that case \[E_{i,j} = C_{i,*}\cdot D_{*,j}^T.\]

What can we do with it? Lets define the following:
\begin{itemize}
\item $G$ is an $n$-by-$m$ matrix where $G_{i,j}=1$ if actor $i$ was in an episode of the show $j$ (and $0$ otherwise)
\item $H$ be an $m$-by-$p$ matrix where $H_{j,k}=1$ if the show $j$ is available to stream on service $k$ (and $0$ otherwise) 
\end{itemize}


%%%%%%%%%%
\section{Square Matrices}
Square matrices (that is, matrices where $m=n$) come up a lot, 
possibly because of this or vice versa there are several properties and operations that exist only on these. 

In a square matrix $N\in\reals^{n\times n}$, we define the \emph{main diagonal} as the entries where the horizontal and vertical component are equal; 
i.e. $\left\{N_{i,i} \mid 1 \le i \le n\right\}$. 

\paragraph{Symmetry.}
We say a square matrix is \emph{symmetric} if $A=A^T$
(and \textit{anti-symmetric} is $A = -A^T$). 
That is, $A$ is symmetric if it is mirrored across the main diagonal which often happens for things like distance matrices (though not always as we'll see). 
Similarly, it is anti-symmetric if it's mirrored across the \textit{anti-diagonal}.

\paragraph{Trace. }
The \emph{trace} of a matrix $tr(A)$ is the sum of the diagonal elements: \[tr(A) := \sum_{I=1}^n A_{i,i}.\] 
The trace does not change under transpose, and is distributive across sum and scalar product. 

\subsection{Identity Matrix}

The \emph{identity} matrix $I_n \in\reals^{n\times n}$ (sometimes simplified to just $I$ when the size is implied from context) 
is a special symmetric matrix where the main diagonal values are $1$ and all other values are $0$.
\[
\forall I,j \in [1,n]: I_{i,j} = \begin{cases} 1 & i=j\\ 0 & i\ne j\end{cases}
\]
Note, $I_n$ is symmetric and $tr(I_n)=n$.


\section*{Useful References}
Liben-Nowell, ``Connecting Discrete Mathematics and Computer Science, 2e''. \S 2.4\\
Wilder, ``10-606-f23:Lecture 3'' GitHub repository, \url{https://github.com/bwilder0/10606-f23/blob/main/files/notes_linalg.pdf}\\
Kolter, ``Linear Algebra Review and Reference'', \url{https://www.cs.cmu.edu/~zkolter/course/15-884/linalg-review.pdf} \S1.1,2.3,3.1-3.5

\end{document}