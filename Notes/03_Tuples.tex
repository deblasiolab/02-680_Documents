\documentclass[11pt, oneside]{article}   	% use "amsart" instead of "article" for AMSLaTeX format

\usepackage{geometry}                		% See geometry.pdf to learn the layout options. There are lots.
\geometry{letterpaper}                   		% ... or a4paper or a5paper or ... 
%\geometry{landscape}                		% Activate for rotated page geometry
\usepackage[parfill]{parskip}    		% Activate to begin paragraphs with an empty line rather than an indent
\usepackage{graphicx}				% Use pdf, png, jpg, or eps§ with pdflatex; use eps in DVI mode
								% TeX will automatically convert eps --> pdf in pdflatex		
\usepackage{amssymb}
\usepackage{amsmath}
\usepackage{tablefootnote}
\usepackage{multirow}
\usepackage{tabularx}
\renewcommand\tabularxcolumn[1]{m{#1}}

\usepackage{dsfont}
\usepackage[hang,flushmargin]{footmisc}

\usepackage{pgfplots,pgfplotstable}

\makeatletter
\long\def\ifnodedefined#1#2#3{%
    \@ifundefined{pgf@sh@ns@#1}{#3}{#2}%
}

\pgfplotsset{
    discontinuous/.style={
    scatter,
    scatter/@pre marker code/.code={
        \ifnodedefined{marker}{
            \pgfpointdiff{\pgfpointanchor{marker}{center}}%
             {\pgfpoint{0}{0}}%
             \ifdim\pgf@y>0pt
                \tikzset{options/.style={mark=*, fill=white}}
                %\draw [densely dashed] (marker-|0,0) -- (0,0);
                \draw plot [mark=*] coordinates {(marker-|0,0)};
             \else
                \tikzset{options/.style={mark=none}}
             \fi
        }{
            \tikzset{options/.style={mark=none}}        
        }
        \coordinate (marker) at (0,0);
        \begin{scope}[options]
    },
    scatter/@post marker code/.code={\end{scope}}
    }
}

\newcommand*\Eval[3]{\left(#1\middle)\right\rvert_{#2}^{#3}}


\usepackage{arydshln}
\usepackage{mathtools}

%Accessibility issues
\usepackage[a-1b]{pdfx}

%\usepackage{tagpdf}
%\tagpdfsetup{activate,paratagging,interwordspace}
\usepackage{accsupp}
\usepackage{hyperref}
\usepackage{lipsum}
%\usepackage{axessibility}
\usepackage{accessibility}

%SetFonts
\DeclareEmphSequence{\bfseries\itshape}
\DeclareMathOperator {\ints}{\mathbb{Z}}
\DeclareMathOperator {\reals}{\mathbb{R}}
\DeclareMathOperator*{\argmin}{argmin}
\DeclareMathOperator*{\argmax}{argmax}


\renewcommand\labelenumi{(\theenumi)}
\renewcommand*{\thefootnote}{\fnsymbol{footnote}}

\usepackage{tcolorbox}

\newenvironment{aside}[1][Aside]{\begin{tcolorbox}[colback=black!5,colframe=black!75!black,title=#1]}{\end{tcolorbox}}

%SetFonts


\title{Topic 3: Tuples}
\author{02-680: Essentials of Mathematics and Statistics}
%\date{}							% Activate to display a given date or no date

\begin{document}
\maketitle

%%%%%
\section{Tuples}
Unlike sets, \emph{tuples} (also called \emph{sequences} or \emph{lists}) are an \textit{ordered} collections of objects.
Think of the position on chess board (or a 2D plane), a color in RGB, etc. 

When these have small cardinality we can use terms like (ordered) pair [2], triple [3], quadruple [4], or more generically an ``$n$-tuple''.

If we're being precise, we normally use angle brackets (``$\langle$'', ``$\rangle$'') but a lot of times we will be lazy and just use parentheses (``('',``)''); 
But we will always differentiate from sets which use curly brackets (``\{'',``\}'').

\paragraph{Cartesian Product.}
A very useful way to construct \emph{set of} tuples is using the \emph{cartesian product} operator, 
which in essence creates all possible pairs of elements from two sets.
\[
S \times T = \left\{\left\langle x,y\right\rangle \mid x \in S \wedge y\in T\right\}
\]

As an example, lets remember the first two sets from our examples last topic: 
\[
A = \left\{\textnormal{``Welcome''}, \textnormal{``to''}, \textnormal{``02-680''}\right\} \textnormal{ and}
\]\[
B = \left\{x^2 \mid x=2 \vee x=3 \right\}.
\]
In this case the cartesian product is 
\[
A \times B = \left\{ \left\langle\textnormal{``Welcome''},4\right\rangle, \left\langle\textnormal{``to''},4\right\rangle, \left\langle\textnormal{``02-680''},4\right\rangle,
 \left\langle\textnormal{``Welcome''},9\right\rangle, \left\langle\textnormal{``to''},9\right\rangle, \left\langle\textnormal{``02-680''},9\right\rangle \right\}
\]

It doesn't have to be different sets in the cartesian product though, we can have the product with a set and itself.
In fact this is performed so often it has its own notation:
\[
B \times B = B^2 = \left\{ \left\langle 4,4 \right\rangle, \left\langle 4,9 \right\rangle,  \left\langle 9,4 \right\rangle, \left\langle 9,9 \right\rangle \right\}.
\]
This notation also generalizes, so $S^3 = S \times S \times S$, $S^4 = S\times S\times S\times S$ and so on.

\begin{tcolorbox}[colback=blue!5,colframe=blue!75!black,title=Quote from Liben-Newell \S2.4]
An annoying pedantic point: we are being sloppy with notation in [the notation above]; 
we only defined the Cartesian product for two sets, 
so when we write $S\times S\times S$ we “must” mean either $S\times (S\times S)$ or $(S\times S)\times S$. 
We’re going to ignore this issue, and simply write statements like $\langle 0, 1, 1\rangle \in \{0, 1\}\times \{0, 1\} \times \{0, 1\}$---%
even though we ought to instead be writing statements like $\langle 0, \langle1, 1\rangle\rangle \in \{0, 1\}\times (\{0, 1\} \times \{0, 1\})$.
\end{tcolorbox}
 
Notice that order matters in tuples (unlike sets) so $ \left\langle 4,9 \right\rangle \neq \left\langle 9,4 \right\rangle$.

A note about this notation:
Sometimes we want to have a set of tuples of different lengths (remember sets don't need to be over objects of the same type) 
so something like 
\[
B^2 \cup B^3 = \left\{  \begin{matrix}\left\langle 4,4 \right\rangle, \left\langle 4,9 \right\rangle,  \left\langle 9,4 \right\rangle, \left\langle 9,9 \right\rangle,\\
				\left\langle 4,4,4 \right\rangle, \left\langle 4,4,9 \right\rangle,  \left\langle 4,9,4 \right\rangle, \left\langle 4,9,9 \right\rangle,\\
				\left\langle 9,4,4 \right\rangle, \left\langle 9,4,9 \right\rangle,  \left\langle 9,9,4 \right\rangle, \left\langle 9,9,9 \right\rangle \end{matrix}\right\}
\]
If we wanted to enumerate all binary numbers up to 8 digits (while omitting leading 0s):
\[
\{1\} \times \bigcup_{i=1}^7 \left\{0,1\right\}^i
\]
But that leads to the notation we saw last time for strings: $\Sigma^*$.
Sometimes we want the set of all tuples of any length, then we use the \emph{Kleene star} (or Kleene operator); 
for some set S, 
\[
S^* =  \bigcup_{i=0}^\infty S^i.
\]
We will define $S^0 = \langle\rangle$ (the empty tuple) for any $S$, in the case of $\Sigma^0$ we often call it the empty string.

\section*{Useful References}
Liben-Nowell, ``Connecting Discrete Mathematics and Computer Science, 2e''. \S 2.4

\end{document}