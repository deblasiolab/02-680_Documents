\documentclass[11pt, oneside]{article}   	% use "amsart" instead of "article" for AMSLaTeX format

\usepackage{geometry}                		% See geometry.pdf to learn the layout options. There are lots.
\geometry{letterpaper}                   		% ... or a4paper or a5paper or ... 
%\geometry{landscape}                		% Activate for rotated page geometry
\usepackage[parfill]{parskip}    		% Activate to begin paragraphs with an empty line rather than an indent
\usepackage{graphicx}				% Use pdf, png, jpg, or eps§ with pdflatex; use eps in DVI mode
								% TeX will automatically convert eps --> pdf in pdflatex		
\usepackage{amssymb}
\usepackage{amsmath}
\usepackage{tablefootnote}
\usepackage{multirow}
\usepackage{tabularx}
\renewcommand\tabularxcolumn[1]{m{#1}}

\usepackage{dsfont}
\usepackage[hang,flushmargin]{footmisc}

\usepackage{pgfplots,pgfplotstable}

\makeatletter
\long\def\ifnodedefined#1#2#3{%
    \@ifundefined{pgf@sh@ns@#1}{#3}{#2}%
}

\pgfplotsset{
    discontinuous/.style={
    scatter,
    scatter/@pre marker code/.code={
        \ifnodedefined{marker}{
            \pgfpointdiff{\pgfpointanchor{marker}{center}}%
             {\pgfpoint{0}{0}}%
             \ifdim\pgf@y>0pt
                \tikzset{options/.style={mark=*, fill=white}}
                %\draw [densely dashed] (marker-|0,0) -- (0,0);
                \draw plot [mark=*] coordinates {(marker-|0,0)};
             \else
                \tikzset{options/.style={mark=none}}
             \fi
        }{
            \tikzset{options/.style={mark=none}}        
        }
        \coordinate (marker) at (0,0);
        \begin{scope}[options]
    },
    scatter/@post marker code/.code={\end{scope}}
    }
}

\newcommand*\Eval[3]{\left(#1\middle)\right\rvert_{#2}^{#3}}


\usepackage{arydshln}
\usepackage{mathtools}

%Accessibility issues
\usepackage[a-1b]{pdfx}

%\usepackage{tagpdf}
%\tagpdfsetup{activate,paratagging,interwordspace}
\usepackage{accsupp}
\usepackage{hyperref}
\usepackage{lipsum}
%\usepackage{axessibility}
\usepackage{accessibility}

%SetFonts
\DeclareEmphSequence{\bfseries\itshape}
\DeclareMathOperator {\ints}{\mathbb{Z}}
\DeclareMathOperator {\reals}{\mathbb{R}}
\DeclareMathOperator*{\argmin}{argmin}
\DeclareMathOperator*{\argmax}{argmax}


\renewcommand\labelenumi{(\theenumi)}
\renewcommand*{\thefootnote}{\fnsymbol{footnote}}

\usepackage{tcolorbox}

\newenvironment{aside}[1][Aside]{\begin{tcolorbox}[colback=black!5,colframe=black!75!black,title=#1]}{\end{tcolorbox}}


\title{Topic 15: Multiple Random Variables}
\author{02-680: Essentials of Mathematics and Statistics}
%\date{}							% Activate to display a given date or no date

\begin{document}
\maketitle

Sometimes we need to deal with the interactions of more than one random variable. 

Consider tossing a (fair) coin 3 times, and define two random variables: 
\begin{itemize}
\item{$X$} --- the number of heads in the first toss
\item {$Y$} --- the number of heads in all 3 tosses
\end{itemize}

We want to know the joint probability (that is, the probability $p(X=x,Y=y)$): 

\begin{center}
\begin{tabular}{cr||c|c|c|c|c}
	&& \multicolumn{4}{c|}{$Y$}\\
	&& 0 & 1 & 2 & 3\\
\hline\hline
\multirow{2}{*}{$X$} & 0	& 1/8		& 1/4		& 1/8		& 0	\\%&	1/2\\
\cline{3-6}
			& 1	&	0	& 1/8		& 1/4		& 1/8		\\%&1/2\\
\hline
%			&	&	1/8	&3/8 		& 3/8		& 1/8 	&\\
\end{tabular}
\end{center}

\section{Marginal Probabilities}
What if we're given the joint probability of some events and want to find out the underlying probability of the events. 

\paragraph{Example.}
Assume we have two \emph{unfair} coins, which we will model as random variables $X$ and $Y$, 

and we're given the following joint probabilities: 
\begin{center}
\begin{tabular}{cr||c|c|c}
	&& \multicolumn{2}{c|}{$Y$}\\
	&& \texttt{Heads} & \texttt{Tails}\\
\hline\hline
\multirow{2}{*}{$X$} & \texttt{Heads}	& 1/10		& 2/10		& 3/10\\
\cline{3-4}
			& \texttt{Tails}	&	3/10	& 4/10		& 7/10\\
\hline
			&			&	4/10	& 6/10\\
\end{tabular}
\end{center}

If we want to determine, say $P(X=\texttt{Tails})$, it turns out we can some over all the possibilities of $Y$: 
\[P(X=\texttt{Tails}) = \sum_{y\in\{\texttt{Heads},\texttt{Tails}\}}p(X=\texttt{Tails},Y=y) = \frac{3}{10} + \frac{4}{10} = \frac{7}{10}\]

Doing all the math, both coins are biased toward \texttt{Tails},  the first coin ($X$) more-so. 

\subsection{Continuous Random Variables}
For continuous random variables, this again turns from sums into integrals. 
Assume you're given a joint distribution function $f(x,y)$.
We can, as before, find the joint probability for a range as a (double) integral: 
\[p(a \le X \le b, c \le Y \le d) = \int_a^b \int_c^d f(x,y) dy dx\]

And to get a marginal distribution function 
\[f(x) = \int_{-\infty}^\infty f(x,y) dy\]


\paragraph{Example.} 
Lets say we're drawing points uniformly over the unit square, 
in this case 
\[f(x,y) = \begin{cases} 1 & 0 \le x,y \le 1\\ 0 & \text{otherwise}\end{cases}\]

So if we wan to find 
\[ p\left(X\le \frac{1}{2}, Y \le \frac{1}{2}\right) = \int_0^\frac{1}{2} \int_0^\frac{1}{2} 1 dydx = 
\Eval{\Eval{xy}{0}{\frac{1}{2}}}{0}{\frac{1}{2}} = 
\Eval{x\cdot\frac{1}{2}-x\cdot0}{0}{\frac{1}{2}}\]\[ = 
\left(\frac{1}{2}\cdot\frac{1}{2} - \frac{1}{2}\cdot0\right) - \left(0 \cdot\frac{1}{2} - 0 \cdot 0\right) = \frac{1}{4}\]

%%%%%%%%%%%%%%
\section{Independence of Random Variables}
Similar to independence for event probabilities, two random variables $X$ and $Y$ are independent iff 
\[
\forall x, y : \;\; p(X=x,Y=y) = p(X=x)\cdot p(Y=y)
\]


Work left to the reader: one of the situations below  describes independent random variables the other does not. 

\begin{center}
\begin{tabular}{cr||c|c|c}
	&& \multicolumn{2}{c|}{$Y$}\\
	&& \texttt{Heads} & \texttt{Tails}\\
\hline\hline
\multirow{2}{*}{$X$} & \texttt{Heads}	& 1/4		& 1/4		& 1/2\\
\cline{3-4}
			& \texttt{Tails}	&	1/4	& 1/4		& 1/2\\
\hline
			&			&	1/2	& 1/2\\
\end{tabular}\hspace{3em}
%
\begin{tabular}{cr||c|c|c}
	&& \multicolumn{2}{c|}{$Y$}\\
	&& \texttt{Heads} & \texttt{Tails}\\
\hline\hline
\multirow{2}{*}{$X$} & \texttt{Heads}	& 1/2		& 0		& 1/2\\
\cline{3-4}
			& \texttt{Tails}	&	0	& 1/2		& 1/2\\
\hline
			&			&	1/2	& 1/2\\
\end{tabular}
\end{center}



%%%%%%%%%%%%%%
\section{Conditional Distributions}
Similar to independence for event probabilities, for two random variables $X$ and $Y$ :
\[p(X\mid Y) = \frac{p(X,Y)}{p(Y)}\]

\begin{center}
\begin{tabular}{c|c|c|c}
\textit{BRCA1} mutation & \textit{P53} mutation & develops cancer \\
$G_1$ & $G_2$ & $C$& $p(G_1,G_2,C)$\\
\hline\hline
\texttt{N}& \texttt{N}& \texttt{N}& $0.24$\\
\texttt{N}& \texttt{N}& \texttt{Y}& $0.01$\\
\texttt{N}& \texttt{Y}& \texttt{N}& $0.20$\\
\texttt{N}& \texttt{Y}& \texttt{Y}& $0.05$\\
\texttt{Y}& \texttt{N}& \texttt{N}& $0.15$\\
\texttt{Y}& \texttt{N}& \texttt{Y}& $0.10$\\
\texttt{Y}& \texttt{Y}& \texttt{N}& $0.05$\\
\texttt{Y}& \texttt{Y}& \texttt{Y}& $0.20$\\
\end{tabular}
\end{center}

Compute $p(C\mid G_1,G_2)$.

\[
p(C\mid G_1,G_2) = \frac{p(G_1,G_2,C)}{p(G_1,G_2)} =  \frac{p(G_1,G_2,C)}{p(G_1,G_2,C=\texttt{N})+p(G_1,G_2,C=\texttt{Y})} 
\]

Note that \[p(G_1=a,G_2=b) = \frac{\sum_{c\in\{\texttt{Y},\texttt{N}\}}p(G_1=a,G_2=b,C=c)}{\sum_{g_1\in\{\texttt{Y},\texttt{N}\}}\sum_{g_2\in\{\texttt{Y},\texttt{N}\}}\sum_{c\in\{\texttt{Y},\texttt{N}\}}p(G_1=g_1,G_2=g_2,C=c)}\]
but because the denominator is $1$ it is left out of the equation above for ease exposition. 

\begin{center}
\begin{tabular}{c|c|c}
%\textit{BRCA1} mutation & \textit{P53} mutation \\
$G_1$ & $G_2$ & $p(C=1\mid G_1,G_2)$\\
\hline\hline
\texttt{N}& \texttt{N}& $\frac{0.01}{0.24+0.01} = 0.04$\\
\texttt{N}& \texttt{Y}& $\frac{0.05}{0.20+0.05} = 0.20$\\
\texttt{Y}& \texttt{N}& $\frac{0.10}{0.15+0.10} = 0.40$\\
\texttt{Y}& \texttt{Y}& $\frac{0.20}{0.05+0.20} = 0.80$\\
\end{tabular}
\end{center}


\section*{Useful References}
Wasserman. ``All of Statistics: A Concise Course in Statistical Inference'' \S\S 2.5-2.7

\end{document}