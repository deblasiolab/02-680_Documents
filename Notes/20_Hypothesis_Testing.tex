\documentclass[11pt, oneside]{article}   	% use "amsart" instead of "article" for AMSLaTeX format

\usepackage{geometry}                		% See geometry.pdf to learn the layout options. There are lots.
\geometry{letterpaper}                   		% ... or a4paper or a5paper or ... 
%\geometry{landscape}                		% Activate for rotated page geometry
\usepackage[parfill]{parskip}    		% Activate to begin paragraphs with an empty line rather than an indent
\usepackage{graphicx}				% Use pdf, png, jpg, or eps§ with pdflatex; use eps in DVI mode
								% TeX will automatically convert eps --> pdf in pdflatex		
\usepackage{amssymb}
\usepackage{amsmath}
\usepackage{tablefootnote}
\usepackage{multirow}
\usepackage{tabularx}
\renewcommand\tabularxcolumn[1]{m{#1}}

\usepackage{dsfont}
\usepackage[hang,flushmargin]{footmisc}

\usepackage{pgfplots,pgfplotstable}

\makeatletter
\long\def\ifnodedefined#1#2#3{%
    \@ifundefined{pgf@sh@ns@#1}{#3}{#2}%
}

\pgfplotsset{
    discontinuous/.style={
    scatter,
    scatter/@pre marker code/.code={
        \ifnodedefined{marker}{
            \pgfpointdiff{\pgfpointanchor{marker}{center}}%
             {\pgfpoint{0}{0}}%
             \ifdim\pgf@y>0pt
                \tikzset{options/.style={mark=*, fill=white}}
                %\draw [densely dashed] (marker-|0,0) -- (0,0);
                \draw plot [mark=*] coordinates {(marker-|0,0)};
             \else
                \tikzset{options/.style={mark=none}}
             \fi
        }{
            \tikzset{options/.style={mark=none}}        
        }
        \coordinate (marker) at (0,0);
        \begin{scope}[options]
    },
    scatter/@post marker code/.code={\end{scope}}
    }
}

\newcommand*\Eval[3]{\left(#1\middle)\right\rvert_{#2}^{#3}}


\usepackage{arydshln}
\usepackage{mathtools}

%Accessibility issues
\usepackage[a-1b]{pdfx}

%\usepackage{tagpdf}
%\tagpdfsetup{activate,paratagging,interwordspace}
\usepackage{accsupp}
\usepackage{hyperref}
\usepackage{lipsum}
%\usepackage{axessibility}
\usepackage{accessibility}

%SetFonts
\DeclareEmphSequence{\bfseries\itshape}
\DeclareMathOperator {\ints}{\mathbb{Z}}
\DeclareMathOperator {\reals}{\mathbb{R}}
\DeclareMathOperator*{\argmin}{argmin}
\DeclareMathOperator*{\argmax}{argmax}


\renewcommand\labelenumi{(\theenumi)}
\renewcommand*{\thefootnote}{\fnsymbol{footnote}}

\usepackage{tcolorbox}

\newenvironment{aside}[1][Aside]{\begin{tcolorbox}[colback=black!5,colframe=black!75!black,title=#1]}{\end{tcolorbox}}


\title{Topic 20: Hypothesis Testing}
\author{02-680: Essentials of Mathematics and Statistics}
%\date{}							% Activate to display a given date or no date

\begin{document}
\maketitle

Once we have taken our data and tried to fit a model to it, 
we often want to know if that fit model matches our original assumptions about the data. 
We call this \emph{hypothesis testing}, 
and we use these techniques to put an actual value on this match. 

We add some terminology on what we've been talking about with respect to statistics 
and frame things as follows: 
we first define our \emph{alternate hypothesis}, which we will denote $H_1$, 
this will be the set of events that define what we want to ask about the confidence in happening; 
we then define the \emph{null hypothesis}, which we denote $H_0$, 
this is the set of events that are all outcomes other than than our alternate. 
%So if we're asking about a coin being unfair, the alternate would be that its biased and the null hypothesis is that its fair. 
So lets say were asking if a drug has a measurable impact on cholesterol, 
the null hypothesis would be that cholesterol stayed the same 
and the alternate hypothesis would be that it changed. 

We usually refer to a hypothesis test telling us if we should \emph{reject} or \emph{retain} the null hypothesis. 

%%%%%%%%%%%
\section{Defining Errors}
Errors occur when the hypothesis test tells us something thats wrong. 
So in the example above 
if the test tells us to reject the null (that is, its confident that the cholesterol changed) 
but in reality it didn't change we call this a \emph{Type I} error. 
On the other hand if our test tells us to retain the null but in reality the value \textit{did} change 
we call that a \emph{Type II} error. 

\begin{center}
\begin{tabular}{|cc||c|c|}
\hline
&& \multicolumn{2}{c|}{Hypothesis Test Result}\\
&& Retain $H_0$ & Reject $H_0$\\
\hline \hline
\multirow{2}{*}{Truth} & $H_0$ & & Type I Error\\
\cline{2-4}
& $H_1$ & Type II Error & \\
\hline
\end{tabular}
\end{center}

We say the 
\begin{itemize}
\item[] \textbf{Type I Error Rate} is $p(\text{reject }H_0, H_0 \text{ is true})$,  
\item[] \textbf{Type II Error Rate} is $p(\text{retain }H_0, H_1 \text{ is true})$, and 
\item[] Statistical \emph{Power} is 1 - Type II Error Rate. 
\end{itemize}
The last point means that the higher power tests have a stronger ability to detect signals for $H_1$. 

Let's look at it visually, first for what we call a \emph{two-sided} test, 
that is $H_0: \mu = x$ and $H_1: \mu \ne x$. 
\begin{center}
\begin{tikzpicture}[
    declare function={gaus(\k,\m,\s)=1/(\s*sqrt(2*pi))*exp(-((\k-\m)^2)/(2*\s^2));}
]
\begin{axis}[
     width=.8\textwidth,
     height=.3\textwidth,
    xmin=-3.5,xmax=6.5,
    xlabel={$\omega$},
    ylabel={$p(\omega)$},
    samples=20,smooth,
    xtick={0,3},
    xticklabels={$\mu_0$,$\mu^*$},
    ytick={0},yticklabels={},
]

\addplot [draw=none, fill=blue!25, domain=1:6.5] {gaus(x, 0, 1)} \closedcycle;
\addplot [draw=none, fill=blue!25, domain=-3.5:-1] {gaus(x, 0, 1)} \closedcycle;
\addplot [blue,domain=-3.5:6.5] {gaus(x,0,1)}; \addlegendentry{$H_0$}
\addplot [draw=none, fill=red!25, domain=-1:1] {gaus(x, 3, 1)} \closedcycle;
\addplot [red,domain=-3.5:6.5] {gaus(x,3,1)}; \addlegendentry{$H_1$}
\draw[ultra thin, dashed] (axis cs:0,0) -- (axis cs:0,0.4);
\draw[ultra thin, dashed] (axis cs:3,0) -- (axis cs:3,0.4);
\draw[ultra thick, green!40!gray] (axis cs:-1,0) -- (axis cs:-1,0.25);
\draw[ultra thick, green!40!gray] (axis cs:1,0) -- (axis cs:1,0.25);
\end{axis}
\end{tikzpicture}
\end{center}

In the figure above, when we pick some boundary around our desired $x$ (the green lines) we will have some probability of 
Type I Error (blue shaded regions) and Type II Error (red shaded region). 

In a \emph{one-sided} test we say 
$H_0: \mu \le x$ and $H_1: \mu > x$.

\begin{center}
\begin{tikzpicture}[
    declare function={gaus(\k,\m,\s)=1/(\s*sqrt(2*pi))*exp(-((\k-\m)^2)/(2*\s^2));}
]
\begin{axis}[
     width=.8\textwidth,
     height=.3\textwidth,
    xmin=-3.5,xmax=6.5,
    xlabel={$\omega$},
    ylabel={$p(\omega)$},
    samples=20,smooth,
    xtick={0,3},
    xticklabels={$\mu_0$,$\mu^*$},
    ytick={0},yticklabels={},
]


\addplot [draw=none, fill=blue!25, domain=1:6.5] {gaus(x, 0, 1)} \closedcycle;
\addplot [blue,domain=-3.5:6.5] {gaus(x,0,1)}; \addlegendentry{$H_0$}
\addplot [draw=none, fill=red!25, domain=-1:1] {gaus(x, 3, 1)} \closedcycle;
\addplot [red,domain=-3.5:6.5] {gaus(x,3,1)}; \addlegendentry{$H_1$}
\addplot [draw=none, pattern color=green!40!gray, domain=1:6.5, pattern=north east lines] {gaus(x, 3, 1)} \closedcycle;
\draw[ultra thin, dashed] (axis cs:0,0) -- (axis cs:0,0.4);
\draw[ultra thin, dashed] (axis cs:3,0) -- (axis cs:3,0.4);
\draw[ultra thick, green!40!gray] (axis cs:1,0) -- (axis cs:1,0.25);
\end{axis}
\end{tikzpicture}
\end{center}

In this example, one again the Type I and Type II errors are shown in the blue and red shaded regions, 
but we can also see the power of the test (green stripped region). 

\paragraph{Main Takeaways. }
In both cases we can choose the cutoff (the thick green lines) of where to make the distinction between $H_0$ and $H_1$, 
but there is a tradeoff: 
\textbf{as Type I error goes down, Type II will go up} and power will go down. 

Similarly, as $\mu^*$ and $\mu_0$ become further apart both errors will go down, and the signal becomes easier to detect. 

%%%%%%%%%%%%%%%%%%%
\section{Performing Tests}

Lets assume we have some data $\mathcal{D} = X_1,X_2,\cdots,X_n$ that follows the same distribution with known $p(X_i\mid\theta)$.
We will perform a test in 3 steps:
\begin{enumerate}
\item Compute a test statistic (a function of the data) thats appropriate for the distribution: \[T = r(X_1,X_2,\cdots,X_n),\]
\item Compute a $p$-value (we will talk about this below), then 
\item For a desired significance level $\beta$ and the $p$-value, decide whether to retain or reject $H_0$. 
\end{enumerate}

We will use the ongoing example from above about testing the efficacy of a drug for high cholesterol, 
lets assume we know the typical variance of cholesterol among humans ($\sigma$) and that this is not going to change between the conditions. 

\paragraph{Step 1. }
In this case, since we're assuming that the samples are coming from a Gaussian (Normal) distribution ($X_1,X_2,\cdots,X_n\sim\mathcal{N}(\mu_0,\sigma^2)$), 
the test statistic is the mean ($\overline{X_n}$). 
And lets assume we're running a two-sided test 
(we don't know if the impact is going to lower or raise cholesterol we just want to know if it changes); thus
$H_0: \mu = \mu_0$ and $H_1: \mu \ne  \mu_0$.

Under the null hypothesis \[\overline{X_n}\sim\mathcal{N}(x, \sigma^2/n)\] 
or as we saw \[\frac{\overline{X_n}-\mu_0}{\sigma/\sqrt{n}}\sim\mathcal{N}(0,1).\]

\paragraph{Step 2.}
We want to compute the $p$-value, which is essentially the probability that we would see the test statistic under the null hypothesis (for $T$):
\[p\left(\left|T\right| > \frac{\overline{X_n}-\mu_0}{\sigma/\sqrt{n}}\right)\]

\paragraph{Step 3.}
Depending on how strict we want to be, we accept or reject $H_0$ based on the $p$-value. 
Some general rules on how to chose $\alpha$:
\begin{center}\begin{tabular}{|l||l|}
\hline
$p$-value & interpretation\\
\hline\hline
$< 0.01$ & very strong evidence against $H_0$\\
$0.01-0.05$&strong evidence against $H_0$\\
$0.05-0.1$&weak evidence against $H_0$\\
$> 0.1$&little to no evidence against $H_0$\\
\hline
\end{tabular}\end{center}

\paragraph{One-sided tests.} 
For a one-sided test; that is one where for example $H_0: \mu \le \mu_0$ and $H_1: \mu > \mu_0$; 
the only thing that changes is that in Step 2: 
\[p\left(T > \frac{\overline{X_n}-\mu_0}{\sigma/\sqrt{n}}\right).\]
Notice that usually one-sided tests can be more powerful as they are only integrating over one region. 

\paragraph{Main Takeaways.}
When performing tests, the main thing we need to do is determine: 
\begin{itemize}
\item the distribution we think the data came from, 
\item the test statistic thats appropriate for those samples, and 
\item the distribution that applies to the test statistic\\ (it may be different than the one for the data). 
\end{itemize}

%%%%%%%%%%%%%
\section{$t$-tests: When $\sigma$ is Unknown}
Lets again assume $X_1,X_2,\cdots,X_n\sim\mathcal{N}(\mu_0,\sigma^2)$ but this time we don't know $\sigma$. 
We're going to define two statistics: 
\[\overline{X_n} = \frac{1}{n}\sum X_i\]
and
\[\overline{\sigma}^2 = \frac{1}{n-1}\sum(X_i-\overline{X_n})^2\]

We're then going to say the following:
\[\frac{\overline{X_n}-\mu_0}{\overline{\sigma}/\sqrt{n}}\sim t_{n-1}\]

\begin{aside}[The $t$ distribution]
The $t$ distribution has one parameter: $\nu$, which is the number of \textit{degrees of freedom}. 
\[X\sim t_\nu\]
\[p(X=x) = \frac{\Gamma\left(\frac{\nu+1}{2}\right)}{\sqrt{\pi\nu}\cdot\Gamma\left(\frac{\nu}{2}\right)}\left(1+\frac{x^2}{\nu}\right)^{-\frac{\nu+1}{2}}\]

The shape of $t$ distribution is similar to the shape of normal
distribution but $t$ distribution has a thicker tail.

\begin{center}
\begin{tikzpicture}[
    declare function={gaus(\k,\m,\s)=1/(\s*sqrt(2*pi))*exp(-((\k-\m)^2)/(2*\s^2));},
    declare function={gamma(\z)=2.506628274631*sqrt(1/\z)+ 0.20888568*(1/\z)^(1.5)+ 0.00870357*(1/\z)^(2.5)- (174.2106599*(1/\z)^(3.5))/25920- (715.6423511*(1/\z)^(4.5))/1244160)*exp((-ln(1/\z)-1)*\z;},
    %declare function={gamma(\x)=(\x-1)!;},
    %declare function={tdist(\x,\v)=(gamma((\v+1.)/2.)/(((pi*\v)^0.5)*gamma(\v/2.)))*(((1.+(\x^2./\v))^(-(\v+1.)/2.));}
    declare function={tdist(\x,\n)= gamma((\n+1)/2.)/(sqrt(\n*pi) *gamma(\n/2.)) *((1+(\x*\x)/\n)^(-(\n+1)/2.));}
]
\begin{axis}[
     width=.8\textwidth,
     height=.3\textwidth,
    xmin=-5,xmax=5,
    xlabel={$\omega$},
    ylabel={$p(\omega)$},
    samples=50,smooth,
    xtick={0},
    xticklabels={$0$},
    ytick={0},yticklabels={},
]
\addplot [blue,domain=-5:5] {gaus(x,0,1)}; \addlegendentry{$\mathcal{N}$}
\addplot [red,domain=-5:5] {tdist(x,1)}; \addlegendentry{$t_1$}
\addplot [orange,domain=-5:5] {tdist(x,5)}; \addlegendentry{$t_{5}$}
\end{axis}
\end{tikzpicture}
\end{center}

As $v\rightarrow\infty$, the distribution above becomes a normal distribution.

\end{aside}

So we have a way to compute something that should be in $t_{n-1}$, 
that is, we assume there is one less degree of freedom than there are elements in the observation. 

Since the probability of the $t$ distribution is difficult to calculate, we typically have lookup tables for it (see below). 
To use one of these tables find your degrees of freedom in the left column and use that row
to find the column with next smaller number from your statistic.
Read the probability in the top row. 
Since your $t$ will probably be a little bit bigger than the value in the table, 
your $p$ will be smaller, eg., $p<0.01$.
If your $t$ is to the right of all numbers, then $p$ is less than the right most probability. 


%%%%%%%%%%%%%%%
\section{Paired Data: Paired $t$-tests}
Many times the data we have is a set of samples measured in two different conditions. 
As an example, a set of patients measured before an after treatment. 
In that case we want to know if the treatment made a consistent change across the population. 
We also don't know where each of the individuals sits compared with some (unknown) $\mu$. 

In these cases our hypotheses are: 
\[H_0: X_1 = X_2 \;\;\;\;\text{ and }\;\;\;\;\;H_1: X_1\ne X_2\]
(where $X_1$ and $X_2$ are the two experimental conditions). 
Another way to say this is to define some $\delta=X_1-X_2$, and let the hypotheses be 
\[H_0: \delta=0 \;\;\;\;\text{ and }\;\;\;\;\;H_1: \delta \ne 0.\]

As we did before we can compute $\overline{\delta_n}$ and $\overline{\sigma}_\delta$ from the data. 
and let our statistic be 
\[\frac{\overline{\delta_n}-\mu_0}{\overline{\sigma_\delta}/\sqrt{n}}\sim t_{n-1}\]


%%%%%%%%%%%%
\section{Testing Categorical Data: $\chi^2$ tests}
Sometimes we have data where we have some underlying conceptual probabilities for a group of categories, 
and want to know how well what we observed fits this concept. 

Specifically, assume we have some underlying assumption that we will see a set of $n$ categories with the following probabilities:
$\dot{p} = (\dot{p}_1, \dot{p}_2, \cdots, \dot{p}_n)$. 
And a set of observations, which we converted to probabilities ${p} = ({p}_1, {p}_2, \cdots, {p}_n)$.
We then want to test  
\[H_0: \dot{p}=p \;\;\;\;\text{ and }\;\;\;\;\;H_1: \dot{p} \ne p\]
(in both cases we assume $\sum \dot{p}_i=1$ and $\sum p_i =1$.)

A good example of this is Mendel’s pea experiment. Mendel’s hypothesis was that the
proportion of round/yellow peas, wrinkled/yellow peas,
round/green peas, and wrinkled/green peas is given as
\[\dot{p} = \left(\frac{9}{16},\frac{3}{16},\frac{3}{16},\frac{1}{16}\right)\]

Lets assume the observations were as follows: 
\begin{center}
\begin{tabular}{|c|c|c|c|c||c|}
\hline
&round/yellow & wrinkled/yellow & round/green & wrinkled/green & total\\
\hline\hline
count & 315 & 101 & 108 & 32 & 556\\
expected counts& 312.75 & 104.25 & 104.25 & 34.75 & 556\\
\hline 
\end{tabular}
\end{center}

In this case the test statistic will be 
\[\sum\frac{\left(c_i - n\cdot \dot{p}_i\right)^2}{ \dot{p}_i)}\sim\chi^2_{n-1}\]

\begin{aside}[The $\chi^2$ distribution]
\[X\sim\chi^2_{p}\]
\[p(X=x) = \frac{x^{\frac{p}{2}-1}}{\Gamma\left(\frac{p}{2}\right)2^{\frac{p}{2}}}e^{\frac{-x}{2}}\]
Here $p$ is the number of degrees of freedom, and in all cases $x>0$.


 \begin{center}
\begin{tikzpicture}[
    declare function={isint(\x) = (int(\x)==\x);},
    declare function={logof2 = 0.693147180559945;},
    declare function={gamma(\z)=2.506628274631*sqrt(1/\z)+ 0.20888568*(1/\z)^(1.5)+ 0.00870357*(1/\z)^(2.5)- (174.2106599*(1/\z)^(3.5))/25920- (715.6423511*(1/\z)^(4.5))/1244160)*exp((-ln(1/\z)-1)*\z;},
    declare function={chisq(\x,\k)=\k<=0||!isint(\k)?1/0:x<=0?0.0:exp((0.5*\k-1.0)*ln(\x)-0.5*\x-gamma(0.5*\k)-\k*0.5*logof2);},
]
\begin{axis}[
     %xmin=-5,xmax=5,
    xlabel={$\omega$},
    ylabel={$p(\omega)$},
    samples=50,smooth,
]
\addplot [blue,domain=0.1:5] {chisq(x,1)}; \addlegendentry{$\chi^2_1$}
\addplot [green,domain=0.1:5] {chisq(x,2)}; \addlegendentry{$\chi^2_2$}
\addplot [red,domain=0.1:5] {chisq(x,3)}; \addlegendentry{$\chi^2_3$}
\addplot [orange,domain=0.1:5] {chisq(x,4)}; \addlegendentry{$\chi^2_4$}
\addplot [purple,domain=0.1:5] {chisq(x,5)}; \addlegendentry{$\chi^2_5$}
\end{axis}
\end{tikzpicture}
\end{center}
\vspace{1em}

If $Z_1,Z_2,\cdots,Z_p$ are independent standard normal random variables,
then
\[\sum Z_i^2 \sim \chi^2_p\]
\end{aside}

Looking at the previous example with Mendel's pea plants. 
\[
\sum_{i=1}{4}\frac{\left(c_i - n\cdot \dot{p}_i\right)^2}{ \dot{p}_i)}\]\[
= \frac{(315-312.75)^2}{312.75} + 
\frac{(101-104.25)^2}{104.25} + 
\frac{(108-104.25)^2}{104.25} + 
\frac{(32-34.75)^2}{34.75}
%\]\[
\approx0.47\]

Just like with $t$-tests we typically look up the $p$-value for $\chi^2$ tests in a table. 
One is included below. 
So we have 3 degrees of freedom, 
thus we cannot reject the null hypothesis. 

%%%%%%%%%%%%%%%%%%%%%%
\section*{Useful References}
Wasserman. ``All of Statistics: A Concise Course in Statistical Inference'' \S10\\
Degroot and Schervish. ``Probability and Statistics''  \S9
\clearpage 
\begin{center}
\begin{tabular}
      {|r||rrr|rrr|rrr|rr|}
      %{r@{\quad}r@{\,}r@{\,}r@{\,}r@{\,}r@{\,}r@{\,}r@{\,}r@{\,}r@{\,}r@{\,}r}
\multicolumn{12}{c}{$t$ Distribution Lookup} \\
\hline
$\nu$&0.4&0.33&0.25&0.2&0.125&0.1&0.05&0.025&0.01&0.005
     &0.001\\
    \hline\hline
 1&0.325&0.577&1.000&1.376&2.414&3.078&6.314&12.706&31.821&63.657&318.31 \\
 2&0.289&0.500&0.816&1.061&1.604&1.886&2.920&4.303&6.965&9.925&22.327 \\
 3&0.277&0.476&0.765&0.978&1.423&1.638&2.353&3.182&4.541&5.841&10.215 \\
 4&0.271&0.464&0.741&0.941&1.344&1.533&2.132&2.776&3.747&4.604&7.173 \\
 5&0.267&0.457&0.727&0.920&1.301&1.476&2.015&2.571&3.365&4.032&5.893 \\
 \hline
 6&0.265&0.453&0.718&0.906&1.273&1.440&1.943&2.447&3.143&3.707&5.208 \\
 7&0.263&0.449&0.711&0.896&1.254&1.415&1.895&2.365&2.998&3.499&4.785 \\
 8&0.262&0.447&0.706&0.889&1.240&1.397&1.860&2.306&2.896&3.355&4.501 \\
 9&0.261&0.445&0.703&0.883&1.230&1.383&1.833&2.262&2.821&3.250&4.297 \\
10&0.260&0.444&0.700&0.879&1.221&1.372&1.812&2.228&2.764&3.169&4.144 \\
\hline
11&0.260&0.443&0.697&0.876&1.214&1.363&1.796&2.201&2.718&3.106&4.025 \\
12&0.259&0.442&0.695&0.873&1.209&1.356&1.782&2.179&2.681&3.055&3.930 \\
13&0.259&0.441&0.694&0.870&1.204&1.350&1.771&2.160&2.650&3.012&3.852 \\
14&0.258&0.440&0.692&0.868&1.200&1.345&1.761&2.145&2.624&2.977&3.787 \\
15&0.258&0.439&0.691&0.866&1.197&1.341&1.753&2.131&2.602&2.947&3.733 \\
\hline
16&0.258&0.439&0.690&0.865&1.194&1.337&1.746&2.120&2.583&2.921&3.686 \\
17&0.257&0.438&0.689&0.863&1.191&1.333&1.740&2.110&2.567&2.898&3.646 \\
18&0.257&0.438&0.688&0.862&1.189&1.330&1.734&2.101&2.552&2.878&3.610 \\
19&0.257&0.438&0.688&0.861&1.187&1.328&1.729&2.093&2.539&2.861&3.579 \\
20&0.257&0.437&0.687&0.860&1.185&1.325&1.725&2.086&2.528&2.845&3.552 \\
\hline
21&0.257&0.437&0.686&0.859&1.183&1.323&1.721&2.080&2.518&2.831&3.527 \\
22&0.256&0.437&0.686&0.858&1.182&1.321&1.717&2.074&2.508&2.819&3.505 \\
23&0.256&0.436&0.685&0.858&1.180&1.319&1.714&2.069&2.500&2.807&3.485 \\
24&0.256&0.436&0.685&0.857&1.179&1.318&1.711&2.064&2.492&2.797&3.467 \\
25&0.256&0.436&0.684&0.856&1.178&1.316&1.708&2.060&2.485&2.787&3.450 \\
\hline
26&0.256&0.436&0.684&0.856&1.177&1.315&1.706&2.056&2.479&2.779&3.435 \\
27&0.256&0.435&0.684&0.855&1.176&1.314&1.703&2.052&2.473&2.771&3.421 \\
28&0.256&0.435&0.683&0.855&1.175&1.313&1.701&2.048&2.467&2.763&3.408 \\
29&0.256&0.435&0.683&0.854&1.174&1.311&1.699&2.045&2.462&2.756&3.396 \\
30&0.256&0.435&0.683&0.854&1.173&1.310&1.697&2.042&2.457&2.750&3.385 \\
\hline
35&0.255&0.434&0.682&0.852&1.170&1.306&1.690&2.030&2.438&2.724&3.340 \\
40&0.255&0.434&0.681&0.851&1.167&1.303&1.684&2.021&2.423&2.704&3.307 \\
45&0.255&0.434&0.680&0.850&1.165&1.301&1.679&2.014&2.412&2.690&3.281 \\
50&0.255&0.433&0.679&0.849&1.164&1.299&1.676&2.009&2.403&2.678&3.261 \\
55&0.255&0.433&0.679&0.848&1.163&1.297&1.673&2.004&2.396&2.668&3.245 \\
60&0.254&0.433&0.679&0.848&1.162&1.296&1.671&2.000&2.390&2.660&3.232 \\
\hline
$\infty$
  &0.253&0.431&0.674&0.842&1.150&1.282&1.645&1.960&2.326&2.576&3.090\\
 \hline
\end{tabular}
\end{center}

\clearpage
\begin{center}
\begin{tabular}
      {|r||rrr|rrr|rrr|rr|}
\multicolumn{12}{c}{$\chi^2$ $p$value lookup table}\\
\hline
$\nu$&0.4&0.33&0.25&0.2&0.125&0.1&0.05&0.025&0.01&0.005
     &0.001\\
    \hline\hline
1&0.708&0.936&1.323&1.642&2.354&2.706&3.841&5.024&6.635&7.879&10.828\\
2&1.833&2.197&2.773&3.219&4.159&4.605&5.991&7.378&9.210&10.597&13.816\\
3&2.946&3.405&4.108&4.642&5.739&6.251&7.815&9.348&11.345&12.838&16.266\\
4&4.045&4.579&5.385&5.989&7.214&7.779&9.488&11.143&13.277&14.860&18.467\\
5&5.132&5.730&6.626&7.289&8.625&9.236&11.070&12.833&15.086&16.750&20.515\\
\hline
6&6.211&6.867&7.841&8.558&9.992&10.645&12.592&14.449&16.812&18.548&22.458\\
7&7.283&7.992&9.037&9.803&11.326&12.017&14.067&16.013&18.475&20.278&24.322\\
8&8.351&9.107&10.219&11.030&12.636&13.362&15.507&17.535&20.090&21.955&26.125\\
9&9.414&10.215&11.389&12.242&13.926&14.684&16.919&19.023&21.666&23.589
 &27.877\\
10&10.473&11.317&12.549&13.442&15.198&15.987&18.307&20.483&23.209&25.188
  &29.588\\
\hline
11&11.530&12.414&13.701&14.631&16.457&17.275&19.675&21.920&24.725&26.757
  &31.264\\
12&12.584&13.506&14.845&15.812&17.703&18.549&21.026&23.337&26.217&28.300
  &32.910\\
13&13.636&14.595&15.984&16.985&18.939&19.812&22.362&24.736&27.688&29.819
  &34.528\\
14&14.685&15.680&17.117&18.151&20.166&21.064&23.685&26.119&29.141&31.319
  &36.123\\
15&15.733&16.761&18.245&19.311&21.384&22.307&24.996&27.488&30.578&32.801
  &37.697\\
\hline
16&16.780&17.840&19.369&20.465&22.595&23.542&26.296&28.845&32.000&34.267
  &39.252\\
17&17.824&18.917&20.489&21.615&23.799&24.769&27.587&30.191&33.409&35.718
  &40.790\\
18&18.868&19.991&21.605&22.760&24.997&25.989&28.869&31.526&34.805&37.156
  &42.312\\
19&19.910&21.063&22.718&23.900&26.189&27.204&30.144&32.852&36.191&38.582
  &43.820\\
20&20.951&22.133&23.828&25.038&27.376&28.412&31.410&34.170&37.566&39.997
  &45.315\\
\hline
21&21.991&23.201&24.935&26.171&28.559&29.615&32.671&35.479&38.932&41.401
  &46.797\\
22&23.031&24.268&26.039&27.301&29.737&30.813&33.924&36.781&40.289&42.796
  &48.268\\
23&24.069&25.333&27.141&28.429&30.911&32.007&35.172&38.076&41.638&44.181
  &49.728\\
24&25.106&26.397&28.241&29.553&32.081&33.196&36.415&39.364&42.980&45.559
  &51.179\\
25&26.143&27.459&29.339&30.675&33.247&34.382&37.652&40.646&44.314&46.928
  &52.620\\
\hline
26&27.179&28.520&30.435&31.795&34.410&35.563&38.885&41.923&45.642&48.290
  &54.052\\
27&28.214&29.580&31.528&32.912&35.570&36.741&40.113&43.195&46.963&49.645
  &55.476\\
28&29.249&30.639&32.620&34.027&36.727&37.916&41.337&44.461&48.278&50.993
  &56.892\\
29&30.283&31.697&33.711&35.139&37.881&39.087&42.557&45.722&49.588&52.336
  &58.301\\
30&31.316&32.754&34.800&36.250&39.033&40.256&43.773&46.979&50.892&53.672
  &59.703\\
\hline
35&36.475&38.024&40.223&41.778&44.753&46.059&49.802&53.203&57.342&60.275
  &66.619\\
40&41.622&43.275&45.616&47.269&50.424&51.805&55.758&59.342&63.691&66.766
  &73.402\\
45&46.761&48.510&50.985&52.729&56.052&57.505&61.656&65.410&69.957&73.166
  &80.077\\
50&51.892&53.733&56.334&58.164&61.647&63.167&67.505&71.420&76.154&79.490
  &86.661\\
55&57.016&58.945&61.665&63.577&67.211&68.796&73.311&77.380&82.292&85.749
  &93.168\\
60&62.135&64.147&66.981&68.972&72.751&74.397&79.082&83.298&88.379&91.952
  &99.607\\
\hline
\end{tabular}
\end{center}

\end{document}