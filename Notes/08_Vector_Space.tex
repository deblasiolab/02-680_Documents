\documentclass[11pt, oneside]{article}   	% use "amsart" instead of "article" for AMSLaTeX format

\usepackage{geometry}                		% See geometry.pdf to learn the layout options. There are lots.
\geometry{letterpaper}                   		% ... or a4paper or a5paper or ... 
%\geometry{landscape}                		% Activate for rotated page geometry
\usepackage[parfill]{parskip}    		% Activate to begin paragraphs with an empty line rather than an indent
\usepackage{graphicx}				% Use pdf, png, jpg, or eps§ with pdflatex; use eps in DVI mode
								% TeX will automatically convert eps --> pdf in pdflatex		
\usepackage{amssymb}
\usepackage{amsmath}
\usepackage{tablefootnote}
\usepackage{multirow}
\usepackage{tabularx}
\renewcommand\tabularxcolumn[1]{m{#1}}

\usepackage{dsfont}
\usepackage[hang,flushmargin]{footmisc}

\usepackage{pgfplots,pgfplotstable}

\makeatletter
\long\def\ifnodedefined#1#2#3{%
    \@ifundefined{pgf@sh@ns@#1}{#3}{#2}%
}

\pgfplotsset{
    discontinuous/.style={
    scatter,
    scatter/@pre marker code/.code={
        \ifnodedefined{marker}{
            \pgfpointdiff{\pgfpointanchor{marker}{center}}%
             {\pgfpoint{0}{0}}%
             \ifdim\pgf@y>0pt
                \tikzset{options/.style={mark=*, fill=white}}
                %\draw [densely dashed] (marker-|0,0) -- (0,0);
                \draw plot [mark=*] coordinates {(marker-|0,0)};
             \else
                \tikzset{options/.style={mark=none}}
             \fi
        }{
            \tikzset{options/.style={mark=none}}        
        }
        \coordinate (marker) at (0,0);
        \begin{scope}[options]
    },
    scatter/@post marker code/.code={\end{scope}}
    }
}

\newcommand*\Eval[3]{\left(#1\middle)\right\rvert_{#2}^{#3}}


\usepackage{arydshln}
\usepackage{mathtools}

%Accessibility issues
\usepackage[a-1b]{pdfx}

%\usepackage{tagpdf}
%\tagpdfsetup{activate,paratagging,interwordspace}
\usepackage{accsupp}
\usepackage{hyperref}
\usepackage{lipsum}
%\usepackage{axessibility}
\usepackage{accessibility}

%SetFonts
\DeclareEmphSequence{\bfseries\itshape}
\DeclareMathOperator {\ints}{\mathbb{Z}}
\DeclareMathOperator {\reals}{\mathbb{R}}
\DeclareMathOperator*{\argmin}{argmin}
\DeclareMathOperator*{\argmax}{argmax}


\renewcommand\labelenumi{(\theenumi)}
\renewcommand*{\thefootnote}{\fnsymbol{footnote}}

\usepackage{tcolorbox}

\newenvironment{aside}[1][Aside]{\begin{tcolorbox}[colback=black!5,colframe=black!75!black,title=#1]}{\end{tcolorbox}}

%SetFonts


\title{Topic 8: Vector Spaces}
\author{02-680: Essentials of Mathematics and Statistics}
%\date{}							% Activate to display a given date or no date

\begin{document}
\maketitle

A \emph{vector space} consists of 3 elements: 
a set of objects $V$ along with definitions of addition and scalar multiplication on the elements in $V$. 
To be considered a vector space, the 3 elements have to satisfy the following conditions: 
\begin{enumerate}
\item $\forall v_1,v_2 \in V : v_1+v_2 \in V$
\item $\forall v_1,v_2 \in V : v_1 + v_2 = v_2 + v_1$
\item $\forall v_1,v_2,v_3 \in V : (v_1 + v_2) + v_3 = v_1 + (v_2 + v_3)$
\item $\exists \textnormal{ unique }\mathbf{0} \in V : \forall v \in V : v + \mathbf{0} = v$ 
\item $\forall v \in V : \exists \textnormal{ unique }u \in V : v + u = \mathbf{0}$ (usually denoted $-v$)
\item $\forall v\in V, \alpha \in \reals: \alpha v \in V$
\item $\forall v_1,v_2\in V, \alpha \in \reals: \alpha (v_1 + v_2) = \alpha v_1 + \alpha v_2$
\item $\forall v_1,v_2\in V, \alpha,\beta \in \reals: (\alpha+\beta) v = \alpha v + \beta v$
\item $\forall v_1,v_2\in V, \alpha,\beta \in \reals: \alpha(\beta v) = (\alpha\beta) v$
\item$\forall v \in V : 1v = v$ 
\end{enumerate} 

\paragraph{Example.} 
Lets ask if the following is a vector space: 
\[V = \left\{\left\langle x_1, x_2\right\rangle \mid x_1,x_2 \in \reals \right\}\]
with addition as 
\[ \langle a_1,a_2\rangle + \langle b_1, b_2\rangle = \langle a_1+b_1, a_2-b_2\rangle\]
(note the second dimension)
and scalar multiplication as 
\[\alpha \langle a_1,a_2\rangle = \langle \alpha a_1, \alpha a_2\rangle.\]

Since multiplication is as we normally see it, we know axioms (6), (8), (9), and (10) are satisfied. 
We can also see that axiom (1) is satisfied.
Let's check (2): 
\[\begin{aligned}
\langle a_1,a_2\rangle + \langle b_1, b_2\rangle & \stackrel{?}{=}  \langle b_1, b_2\rangle + \langle a_1,a_2\rangle\\
\langle a_1+b_1, a_2-b_2\rangle & \stackrel{?}{=}  \langle b_1+a_1, b_2-a_2\rangle\\
a_1+b_1 & =  b_1+a_1\\
\color{red} a_2-b_2& \color{red}\ne  b_2-a_2\\
\end{aligned}\]

Therefore, what we defined (with the non-standard addition) is not a vector space. 


\paragraph{Other Spaces.} 
That said, with standard addition $\reals^2$ \emph{is} a vector space (I will leave it to you to verify).
The most common vector spaces we will be using in this class is $\reals^n$ (for a fixed $n$).

In addition to $\reals^n$, $\reals^{m\times n}$ are also vector spaces (for fixed $m$ and $n$)
with the usual definitions of addition and scalar multiplication for matrices developed last week. 
(Kind of silly since $\reals^{m\times n}$ are matrices, not vectors but thats okay.)

The set of real-valued functions $F$ (over a fixed interval) is also a vector space, 
though in this case we may call it a \emph{function space}. 
We can define 
\[(f+g)(x) = f(x) + g(x) \;\;\;\textnormal{and}\;\;\; (\alpha f)(x) = \alpha f(x).\]
The proof of this is left to you. 

\section{Subspaces}
A subset $S$ of a vector space $V$ is called a \emph{subspace} if $S$ is also a vector space \textit{under the operations inherited from $V$ } (note this would inherently require that $\mathbf{0}\in S$, but this should fall out from the requirements below).

Every vector space has at least two subspaces: (1) $V$ itself, and (2) $\{\mathbf{0}\}\subseteq V$. 
These are both called \textit{trivial} subspaces. 

\paragraph{Examples.} 
Lets look at two subsets of $\reals^2$:
\[S = \left\{\left\langle 0, x_2\right\rangle \mid x_2 \in \reals\right\} \;\;\;\textnormal{and}\;\;\;
T = \left\{\left\langle 1, x_2\right\rangle \mid x_2 \in \reals\right\}\]
Are either of these subspaces? 

We know both $S, T \subseteq \reals^2$, so really we need to check if $S$ and $T$ are vector spaces. 

\paragraph{Lets look at $S$ first: }
because any real number times 0 is 0, as well as $0+0=0$, we can see that all of the axioms above hold. 
The first dimension always remains $0$, and the second dimension inherits all of it's properties from scalar addition and multiplication. 

\paragraph{What about $T$? }
We can start with axiom (1): 
for $a,b \in T$
\[\begin{aligned}
a + b & \stackrel{?}{\in}  T\\
\langle 1,a_2\rangle + \langle 1, b_2\rangle & \stackrel{?}{\in}  T\\
\langle 1+1,a_2+b_2\rangle & \stackrel{?}{\in}  T\\
\color{red} \langle 2,a_2+b_2\rangle& \color{red}\notin T\\
\end{aligned}\]
I will leave it to you to show that several other axioms do not hold (namely axiom (6)). 

\subsection{Proving Subspaces}
For any non-empty subset $S\subseteq V$ for vector space $V$. 
$S$ is a subspace iff it is closed under addition and scalar multiplication. 

That is, for any subset (thats not empty), if we know $V$ is a vector space, we only need to prove (1) and (6) to show $S$ is a subspace. 

\paragraph{Example.}
Lets define \[P_4 = \left\{a_4x^4 + a_3x^3 + a_2x^2 + a_1x^1 + a_0 \mid a_i\in \reals, \forall i \in [4] \right\}.\]
We mentioned (but didn't prove here) functions are a vector space, and clearly $P_4 \subseteq F$. 
To show $P_4$ is a subspace we only need to show that (1) and (6) hold. 

For $a,b \in P_4$ and $\alpha\in\reals$:
\[\begin{aligned} (a_4x^4 + a_3x^3 + a_2x^2 + a_1x^1 + a_0) + (b_4x^4 + b_3x^3 + b_2x^2 + b_1x^1 + b_0) & \stackrel{?}{\in}  P_4\\
\color{blue}(a_4+b_4)x^4 + (a_3+b_3)x^3 + (a_2+b_2)x^2 + (a_1+b_1)x^1 + (a_0+b_0)  &\color{blue}\in  P_4\end{aligned}\]
and
\[\begin{aligned} \alpha(a_4x^4 + a_3x^3 + a_2x^2 + a_1x^1 + a_0) & \stackrel{?}{\in}  P_4\\
\color{blue}(\alpha a_4)x^4 + (\alpha a_3)x^3 + (\alpha a_2)x^2 + (\alpha a_1)x^1 + (\alpha a_0)  &\color{blue}\in  P_4\end{aligned}.\]
Therefore $P_4$ is a subspace of $F$. 




\section*{Useful References}
Isaak and Monougian, ``Basic Concepts of Linear Algebra''. \S 2.1-2.3\\

\end{document}